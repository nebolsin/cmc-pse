%!TEX root = /Users/nebolsin/Documents/MSU/Graduate Work/tex/main.tex

\begin{abstract}
  В связи с постоянным ростом мощности и сложности распределенных вычислительных систем появляется необходимость в организации удобного доступа к таким системам. Объектом проведенного исследования является деятельность междисцилинарных научных сообществ, состоящих из ученых и инженеров, совместно работающих над решением сложных задач. В данной работе проводится анализ принципов работы и вычислительных потребностей таких сообществ (раздел~\ref{intro}), а также описываются пять существующих систем, созданных для поддержки исследований в различных областях (раздел~\ref{review}). На основании полученных данных выявляются основные требования к распределенным вычислительным средам для междисциплинарных сообществ и разрабатывается высокоуровневая архитектура организации таких систем (раздел \ref{research}). 
  
  В практической части работы (раздел~\ref{practical}) разработанная архитектура применяется для реализации рабочего прототипа распределенной вычислительной системы для обработки данных и исследования параллельных алгоритмов, предназначенного для использования на вычислительных мощностях факультета ВМиК МГУ. Основной особенностью программной реализации является использование современных интернет-технологий, что позволяет предоставить доступ к вычислительной среде с любого компьютера, подключенного к сети интернет.
\end{abstract}


