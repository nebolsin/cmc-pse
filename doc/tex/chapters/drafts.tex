\subsection{Характеристики распределенных вычислительных систем для междисциплинарных сообществ}
\label{characteristics}

\begin{itemize}
  \item Упор на реализацию компонентов, а не на их соединение. Традиционные прокраммные среды делают упор либо на реализацию компонентов, либо на связь компонентов между собой (для моделирования сложных вычислений). Например, когда созданию компонентов уделяется особое внимание, а объединение компонентов реализуется в распределенной объектной системе, для создания новых компонентов могут использоваться такие техники как наследование и шаблоны. В других реализациях, где используется параллельное программирование, новые компоненты могут создаваться с помощью специального API. Если же во главу угла ставится объединение компонентов, то зачастую используется моделирование вычислительного процесса путем графического расположения и соединения набора компонентов. Если имеется достаточно обширная библиотека компонентов-примитивов, акцент перемещается с программирования компонентов на создание их композиций.
  
Архитектурные решения в части реализации компонентов и их объединения соотвтествующим образом влияют на набор вариантов, доступных для программирования и композиции. В междисциплинарных сообществах (см. сценарий 1) от программных сред требуется поддержка обеих концепций в равной степени.     
  \item Понятийное рассогласование между компонентами. Косвенным следствием типичных решений в области композиционного моделирования является привязка ученого к определенному реализацией набору понятий. Например, в LSA (и большинстве объектно-ориентированных реализаций) компоненты должны быть высокопроизводительными C++ объектами. Это не является серьезным ограничением для типичных распределенных сообществ, так как в них обычно существуют соглашения на используемые в процессе работы технологии. В данном случае, чтобы бороться с рассогласованием достаточно применения конвертеров форматов и следования стандартам (например, представление матриц в формате CSR и в формате CSC).
  
  В междисциплинарных распределенных сообществах (см. сценарий 1 и сценарий 2) существует огромная разница в наборе понятий у разных ученых (например, биологи, экологи и экономисты, использующие PSE анализа водоразделов оперируют практически не пересекающимися наборами терминов) и фундаментальные различия в понимании правильной организации и моделирования вычислений. Более того, многие компоненты в таких системах уже написаны на разных языках программирования и используют разные форматы и источники данных. Все эти аспекты должны поддерживаться распределенной вычислительной системой. Понятийное рассогласование является серьезной проблемой и делает невозможным создание единого стандарта создания компонентов. 
  
  \item Усложненное управление вычислениями. Традиционные РВС четко различают реализацию компонента и его представление. Преставление обычно используется для именования компонента и описания набора характеристик (например, <<является ли компонент совместимым с gcc-2.7.2?>>). Обычно, это не является серьезным ограничением, так как распределенные сервисы обычно направлены скорее на запуск вычислений, а не на высокоуровневое решение проблем. Сложность управления вычислениями прямо связана с адекватностью представления компонентов в РВС.
  
  В ситуациях, описанных в сценариях 2 и 3, ученые могут сказать: <<Нужно провести те же вычисления, что и в пятницу, но использовать при этом новые результаты, полученные Александром>>. Управление вычислениями включает в себя как высокоуровневые аспекты (запуск имитаций, контроль за выполнением, и т.д.), так и более низкоуровневые (например, подмешивание результатов предыдущих экспериментов на одной из стадий вычислений). Из этого следует, что средства управления данными не должны являться отдельным слоем или сервисом, а должны быть глубоко интегрированы в систему и допускать простое использование на любых этапах работы. Ведение истории проведенных экспериментов и вычислений также является одной из функций системы управления вычисленийми. 
\end{itemize}
