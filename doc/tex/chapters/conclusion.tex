%!TEX root = /Users/nebolsin/Documents/MSU/Graduate Work/tex/main.tex
\section{Заключение}   
\label{conclusion}

В данной работе была последовательно разработана и реализована распределенная вычислительная среда, упрощающая доступ ученых к высокопроизводительным вычислениям. Во введении были выделены ключевые требования к подобным системам, основывающиеся на анализе публикаций в этой области и изучении пяти существующих сред решения задач.

Эти требования были использованы при разработке архитектуры и реализации вычислительной среды, что позволило выделить и описать такие основные сущности как компонент, модель, экземпляр модели и имитация, а также определить аппаратную конфигурацию, необходимую для работы среды.

В рамках работы был реализован и запущен в эксплуатацию прототип распределенной вычислительной среды, предназначенный для использования на вычислительных мощностях факультета ВМиК. Проект работает на следующих серверах:
\begin{itemize}
  \item Основной сервер (angel.cs.msu.su), контролирующий всю работу системы и предоставляющий веб-интерфейс для пользователей.
  \item Вычислительный узел 1 (bluegene1.hpc). IBM Blue Gene/P — массивно-параллельная вычислительная система, которая состоит из двух стоек, включающих 8192 процессорных ядер (2 x 1024 четырехядерных вычислительных узлов), с пиковой производительностью 27,9 терафлопс (27,8528 триллионов операций с плавающей точкой в секунду).   
  \item Вычислительный узел 2 (regatta.hpc). Сервер IBM pSeries 690 Regatta, установленный на факультете --- это крупный сервер масштаба предприятия, состоящий из 16 процессоров Power4 и имеющий 64GB оперативной памяти.
\end{itemize}

Архитектура проекта позволяет легко добавлять новые вычислительные узлы, что увеличит общую производительность системы.

Основными возможностями реализованной вычислительной среды являются:
\begin{itemize}
  \item доступ через веб-интерфейс;
  \item возможность обработки пользовательских данных при помощи уже предустановленных вычислительных компонентов;
  \item возможность создания собственных вычислительных компонентов и автоматическое размещение указанных компонентов на вычислительных узлах;
  \item обеспечение разграничения доступа пользователей к высокопроизводительным вычислительным узлам;
  \item сохранение результатов всех вычислений для обеспечения возможности последующего анализа;
  \item возможность автоматического запуска отдельного компонента с различными параметрами и на различных вычислительных узлах для анализа производительности параллельных алгоритмов на различных высокопроизводительных платформах и конфигурациях.
\end{itemize}