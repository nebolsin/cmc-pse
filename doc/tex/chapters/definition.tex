%!TEX root = /Users/nebolsin/Documents/MSU/Graduate Work/tex/main.tex
\section{Постановка задачи} 
\label{definition}

Необходимо провести исследование существующих систем PSE (таких как WBCSim, VizCraft, L2W, \SW и Expresso), применяемых учеными в различных областях, и выделить существенные принципы работы таких систем. Используя полученные данные в совокопупности с информацией, изложенной во введении, необходимо сформулировать основные требования к PSE для междисциплинарных сообществ.

Далее требуется разработать архитектуру конкретной среды, которая позволит пользователю максимально абстрагироваться от сложных низкоуровневых технологий и сосредоточиться на решении стоящей перед ним задачи. Разработка включает в себя выбор адекватного представления для вычислительных компонентов, проработку взаимодействия между основным сервером и вычислительными узлами, а также создание удобного пользовательского интерфейса.

В практической части работы будет реализована прототипная PSE, работающая на вычислительных мощностях МГУ (таких как BlueGene/P, Regatta, кластре Hill и других).

Ключевые требования к среде:
\begin{itemize}
  \item доступ через веб-интерфейс;
  \item возможность обработки пользовательских данных при помощи уже предустановленных вычислительных компонентов;
  \item возможность создания собственных вычислительных компонентов и автоматическое размещение указанных компонентов на вычислительных узлах;
  \item обеспечение разграничения доступа пользователей к высокопроизводительным вычислительным узлам;
  \item сохранение результатов всех вычислений для обеспечения возможности последующего анализа;
  \item возможность автоматического запуска отдельного компонента с различными параметрами и на различных вычислительных узлах для анализа производительности параллельных алгоритмов на различных высокопроизводительных платформах и конфигурациях.
\end{itemize}

Необходимо развернуть и настроить систему на вычислительных мощностях факультета ВМиК:
\begin{itemize}
  \item Основной сервер (angel.cs.msu.su). Данный сервер будет контролировать всю работу системы и предоставлять веб-интерфейс для пользователей.
  \item Вычислительный узел 1 (bluegene1.hpc). IBM Blue Gene/P — массивно-параллельная вычислительная система, которая состоит из двух стоек, включающих 8192 процессорных ядер (2 x 1024 четырехядерных вычислительных узлов), с пиковой производительностью 27,9 терафлопс (27,8528 триллионов операций с плавающей точкой в секунду).   
  \item Вычислительный узел 2 (regatta.hpc). Сервер IBM pSeries 690 Regatta, установленный на факультете --- это крупный сервер масштаба предприятия, состоящий из 16 процессоров Power4 и имеющий 64GB оперативной памяти.
\end{itemize}

 